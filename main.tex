%%%%%%%%%%%%%%%%%%%%%%%%%%%%%%%%%%%%%%%%%
% Jacobs Landscape Poster
% LaTeX Template
% Version 1.0 (29/03/13)
%
% Created by:
% Computational Physics and Biophysics Group, Jacobs University
% https://teamwork.jacobs-university.de:8443/confluence/display/CoPandBiG/LaTeX+Poster
% 
% Further modified by:
% Nathaniel Johnston (nathaniel@njohnston.ca)
%
% This template has been downloaded from:
% http://www.LaTeXTemplates.com
%
% License:
% CC BY-NC-SA 3.0 (http://creativecommons.org/licenses/by-nc-sa/3.0/)
%
%%%%%%%%%%%%%%%%%%%%%%%%%%%%%%%%%%%%%%%%%

%----------------------------------------------------------------------------------------
% PACKAGES AND OTHER DOCUMENT CONFIGURATIONS
%----------------------------------------------------------------------------------------

\documentclass[final]{beamer}

\usepackage{xparse}      % << to create new commands
\usepackage{tikz}   % << diagrams
  \usetikzlibrary{arrows}
\usepackage{url}

\usepackage[scale=1]{beamerposter} % Use the beamerposter package for laying out the poster
\usetheme{confposter} % Use the confposter theme supplied with this template

\setbeamercolor{block title}{fg=ngreen,bg=white} % Colors of the block titles
\setbeamercolor{block body}{fg=black,bg=white} % Colors of the body of blocks
\setbeamercolor{block alerted title}{fg=white,bg=dblue!70} % Colors of the highlighted block titles
\setbeamercolor{block alerted body}{fg=black,bg=dblue!10} % Colors of the body of highlighted blocks
% Many more colors are available for use in beamerthemeconfposter.sty

%-----------------------------------------------------------
% Define the column widths and overall poster size

% To set effective sepwid, onecolwid and twocolwid values, first choose how many
% columns you want and how much separation you want between columns

% In this template, the separation width chosen is 0.024 of the paper width and
% a 4-column layout onecolwid should therefore be (1-(# of columns+1)*sepwid)/#
% of columns e.g. (1-(4+1)*0.024)/4 = 0.22

% Set twocolwid to be (2*onecolwid)+sepwid = 0.464
% Set threecolwid to be (3*onecolwid)+2*sepwid = 0.708

\newlength{\sepwid}
\newlength{\onecolwid}
\newlength{\twocolwid}
\newlength{\threecolwid}
\setlength{\paperwidth}{48in} % A0 width: 46.8in
\setlength{\paperheight}{36in} % A0 height: 33.1in
\setlength{\sepwid}{0.024\paperwidth} % Separation width (white space) between columns
\setlength{\onecolwid}{0.22\paperwidth} % Width of one column
\setlength{\twocolwid}{0.464\paperwidth} % Width of two columns
\setlength{\threecolwid}{0.708\paperwidth} % Width of three columns
\setlength{\topmargin}{-0.5in} % Reduce the top margin size
%-----------------------------------------------------------

\usepackage{graphicx}  % Required for including images
\usepackage{booktabs} % Top and bottom rules for tables

%-----------------------------------------------------------
% fonts
\usepackage{fontspec} 
\usepackage{xunicode}
\usepackage{xltxtra}
\defaultfontfeatures{Mapping=tex-text}
\setromanfont[Ligatures={Common},Numbers={OldStyle},Variant=01]{Linux Libertine O}
%-----------------------------------------------------------

%----------------------------------------------------------------------------------------
% TITLE SECTION 
%----------------------------------------------------------------------------------------

\title{Multigrid preconditioner for saddle point problems with highly variable coefficients}
\author{Mattia Penati and Edie Miglio}
\institute{Politecnico di Milano - Department of Mathematics - MOX}

%----------------------------------------------------------------------------------------

% ======================================================== math
\usepackage{listings}
\usepackage[ruled]{algorithm2e}
\usepackage{mathtools} % << extends amsmath
\usepackage{amsfonts}  % << some extra fonts
\usepackage[
  bold-style=ISO,
]{unicode-math}
\usepackage{nicefrac}  % << to make nice fraction

\setmathfont{Latin Modern Math}
\setmathfont[range=\mathbb]{TeX Gyre Pagella Math}

\DeclareDocumentCommand\defeq{}{\mathrel{\mathop:}=}
\DeclareDocumentCommand\dual{m}{#1^*}
\DeclareDocumentCommand\adj{m}{#1^T}
\DeclareDocumentCommand\inv{m}{#1^{-1}}
\DeclareMathOperator*{\Kernel}{Ker}
\DeclareMathOperator*{\Range}{Im}

\DeclareMathOperator{\grad}{\nabla}
\AtBeginDocument{
  \let\div\relax
  \DeclareMathOperator{\div}{div}
}
\DeclareDocumentCommand\vec{m o o}{ %
  \mathbf{#1}\IfValueT{#2}{_{#2}}\IfValueT{#3}{^{#3}} %
}
\DeclareDocumentCommand\tensor{m}{#1}

\DeclareDocumentCommand\inner{m m o}{ %
  {\left( #1 , #2 \right)}\IfValueT{#3}{_{#3}} %
}
\DeclareDocumentCommand\norm{m o o}{ %
  {\left\|{#1}\right\|}\IfValueT{#2}{_{#2}}\IfValueT{#3}{^{#3}} %
}
\DeclareDocumentCommand\duality{m m o}{ %
  {\left\langle #1 , #2 \right\rangle}\IfValueT{#3}{_{\dual{#3} \times #3}}
}

\begin{document}

\addtobeamertemplate{block end}{}{\vspace*{2ex}} % White space under blocks
\addtobeamertemplate{block alerted end}{}{\vspace*{2ex}} % White space under highlighted (alert) blocks

\setlength{\belowcaptionskip}{2ex} % White space under figures
\setlength\belowdisplayshortskip{2ex} % White space under equations

\begin{frame}[t] % The whole poster is enclosed in one beamer frame

\begin{columns}[t] % The whole poster consists of three major columns, the second of which is split into two columns twice - the [t] option aligns each column's content to the top

\begin{column}{\onecolwid} % The first column

\begin{alertblock}{Objectives}
The aim of this work is the developement of a robust preconditioning technique
for saddle point problems with highly variable coefficients, especially the
Darcy problem. The idea is based on the augmented Lagrangian formulation of
saddle point problem and the multigrid preconditioning technique in $H(\div)$
developed by Arnold \textit{et al.}. This kind of preconditioner avoids the
construction of Schur complement and all the operators are approxymated by the
means of conforming finite element spaces.
\end{alertblock}

\begin{block}{Introduction}
The Darcy's law describes the flow of fluids through porous media. This
equation is used to model various type of phenomena such as groundwater flows
in aquifers and expulsion of hydrocarbons from the source rock and subsequent
seeping into reservoirs.

Even though the model might look simple, its numerical solution is not trivial,
in particular due of the high variability of the equation's coefficients.
Numerical iterative methods need proper preconditioning in order to find a
solution of the problem in a reasonable time, and research on the best
preconditioning techniques is still ongoing, especially in the context of
parallel programming. This work aims to provide an implementation of an optimal
preconditioner, \textit{i.e.} a preconitioner whose performance does not depend
on the mesh size and on the variations of the equation coefficients. The
advocated method is based on the work of Arnold, Falk and Winther in
\cite{arnold1997preconditioning}.

A block-diagonal preconditioner based on the augmented Lagrangian formulation
of the saddle point problem will be presented. As pointed out by other authors
the use of this formulation instead of the original one allows us to avoid the
necessity to construct a good approximation of the Schur complement and its
inverse. This approach seems to be more natural for the mixed problems embedded
in $H(\div)\times L^2$, like the Darcy problem.
\end{block}

\begin{block}{Mathematical introduction}
Let $V$ be an Hilbert space and denote its inner product with
$\inner{\cdot}{\cdot}[V]$. A linear continuous operator $A:V \to \dual{V}$ will
be considered. Let $Q$ be another Hilbert space and $B:V \to \dual{Q}$ a linear
continuous operator and $\adj{B}:Q \to \dual{V}$ its own adjoint.

Given $\vec{f} \in \dual{V}$ and $g \in \dual{Q}$, the aim is to find
$\vec{u} \in V$ and $p \in Q$ solutions of the problem
\[
  \begin{cases}
    A\vec{u} + \adj{B}p = \vec{f} \quad &\text{in }\dual{V}, \\
    B\vec{u} = g                  \quad &\text{in }\dual{Q},
  \end{cases}
\]
If the operator $A$ is invertible on $\Kernel{B}$ and the operator $B$ is
closed, that is, there exists $\beta > 0$ such that the $\inf$-$\sup$ condition
is satisfied
\[
    \sup_{\vec{v} \in V}
    \frac{\duality{B\vec{u}}{q}[Q]}{\norm{\vec{v}}[V]} \ge
    \beta \norm{q}[Q],
    \qquad
    \forall q \in Q.
\]
For all $\vec{f} \in \dual{V}$ and $g \in \Range{B} \subseteq \dual{Q}$ there
exists a unique solution $\vec{u}\in V$ and $p\in Q/\Kernel{\adj{B}}$ of the
problem.
\end{block}

\end{column} % End of the first column


\begin{column}{\onecolwid} % Begin of the second column

\begin{block}{Augmented Lagrangian-type preconditioners}
If the hypotheses of the existence and uniqueness of solution theorem are
fulfilled and both the operators
\[
    \hat{A}\defeq A + \adj{B}\inv{M}B: V \to \dual{V}
    \qquad\text{and}\qquad
    M:Q \to \dual{Q},
\]
define an equivalent inner product on $V$ and $Q$, then all the eigenvalues of
the generalized problem
\[
    B\inv{\hat{A}}\adj{B}q = \lambda Mq, \qquad\text{in }\dual{Q},
\]
belong to the interval $[\beta^2,1]$.

This result can be used to prove that the following block operators
\[
    \begin{bmatrix}A & B^T \\ B & 0\end{bmatrix}
    \qquad\text{and}\qquad
    \begin{bmatrix}\hat{A} & 0 \\ 0 & M\end{bmatrix},
\]
are spectrally equivalent; the spectrum lies in $[-1, -\beta^2] \cup \{1\}$.

For the Darcy problem the operator $M$ is defined as the identity operator on
$Q$, leading to the following $\hat{A}$ operator
\[
    \hat{A} = \frac{1}{\mu} I_V - \grad \div.
\]
\end{block}

\begin{block}{Multigrid preconditioning}
Solving the preconditioner with a direct method can have serious drawbacks when
the dimension of the problem is large. The two biggest concerns regard poor
scaling and excessive memory consumption. A geometric multigrid preconditioner
has been implemented for our problem, reproducing the work by Arnold, Falk and
Winther in \cite{arnold1997preconditioning}.

This method is based on a standard V-cycle multigrid, but a proper smoother
operator $R_j$ has to be defined on each level $j=1,\dots,J$. The authors
proposed two different smoothers, which are based respectively on an additive
and a multiplicative Schwarz operator.

The additive smoother $R_j : V_j \rightarrow V_j$ is defined as a multiple of
the additive Schwarz operator applied to the following decomposition of $V_j$:
let $\mathcal{N}_j$ be the set of vertices in the triangulation $T_j$; for
every $\nu \in \mathcal{N}_j$, $T_{j, \nu}$ is defined as the patch made of
mesh elements meeting at the vertex $\nu$.
Then we can define our smoother as
\[
    R_j = \eta \sum_{\nu \in \mathcal{N}_j} P_{j, \nu} \; \Lambda_{j}^{-1}.
\]
\end{block}

\begin{block}{Implementation details}
The code is based on the finite element library deal.II
\cite{BangerthHartmannKanschat2007}, version 8.1.0, which
provided us with all the building blocks we needed.
In particular, a module in the library implements the geometric multigrid
method on hierarchical meshes, along with various helper functions.

Thanks to these features we were able to quickly implement the precondioner
and plug it in the deal.II solving procedures.

We aimed at hybrid parallelization using intel TBB library and MPI-enabled
algebraic backends.
MPI support for distributed meshes is still experimental, but we were able
to run scaling tests on the preconditioner initialization procedures.
On the other hand, support for shared memory parallelism is complete.

\end{block}

%----------------------------------------------------------------------------------------

\end{column} % End of the second column


\begin{column}{\onecolwid} % The third column

\begin{block}{Results}

The first problem is defined in $\Omega = (0,\nicefrac{7}{3}) \times (0,1)$ and
it is partitioned into two subdomains $\Omega_1$ and $\Omega_2$. In the first subdomain $\mu$ is taken
constant and equal to $1, 10^{1}, 10^{2}, 10^{4}, 10^{8}$, in the second one
$\mu=1$. The forcing term $g$ is zero and the boundary conditions are chosen in
order to have a flux from left to right:
\begin{itemize}
  \item on the top and the bottom of the domain the normal component of
    $\vec{u}$ is taken equal to zero,
  \item on the right boundary $p=1$ and on the left one $p=0$.
\end{itemize}

\begin{figure}
    \begin{tikzpicture}[scale=2.5,font=\small]
        \draw[fill=white,fill opacity=0.3, very thick] (0,0) rectangle (7,3);
\draw[fill=black, fill opacity=0.3, very thick] (1,0) rectangle (3,2);
\draw[fill=black, fill opacity=0.3, very thick] (4,1) rectangle (6,3);

\node at (3.5,1.5) {$\Omega_1$};
\node at (2,1) {$\Omega_2$};
\node at (5,2) {$\Omega_2$};

\node[left] at (0,1.5) {$p=1$};
\node[right] at (7,1.5) {$p=0$};
\node[above] at (3.5,3) {$\vec{u}\cdot\vec{n}={0}$};
\node[below] at (3.5,0) {$\vec{u}\cdot\vec{n}={0}$};

\begin{scope}[shift={(0,-0.7)}, thick]
  \draw[<->] (0,0) -- (1,0) node[midway,below] {$\nicefrac{1}{3}$};
  \draw[<->] (1,0) -- (3,0) node[midway,below] {$\nicefrac{2}{3}$};
  \draw[<->] (3,0) -- (4,0) node[midway,below] {$\nicefrac{1}{3}$};
  \draw[<->] (4,0) -- (6,0) node[midway,below] {$\nicefrac{2}{3}$};
  \draw[<->] (6,0) -- (7,0) node[midway,below] {$\nicefrac{1}{3}$};
\end{scope}

\begin{scope}[shift={(-1.3,0)}, thick]
  \draw[<->] (0,0) -- (0,1) node[midway,left] {$\nicefrac{1}{3}$};
  \draw[<->] (0,1) -- (0,2) node[midway,left] {$\nicefrac{1}{3}$};
  \draw[<->] (0,2) -- (0,3) node[midway,left] {$\nicefrac{1}{3}$};
\end{scope}



    \end{tikzpicture}
\end{figure}

The number of outer iterations of GMRES method for each numerical experiment of
this case is constant and it does not depend on $h$ and $\mu$.

\begin{table}
    {\small \begin{tabular}{cccccc}
  \toprule
  $h$ & $\mu_1=1$ & $\mu_1=10^1$ & $\mu_1=10^2$ & $\mu_1=10^4$ & $\mu_1=10^8$ \\
  \midrule
  $0.1$     & $2$ & $2$ & $2$ & $2$ & $2$ \\
  $0.05$    & $2$ & $2$ & $2$ & $2$ & $2$ \\
  $0.025$   & $2$ & $2$ & $2$ & $2$ & $2$ \\
  $0.0125$  & $2$ & $2$ & $2$ & $2$ & $2$ \\
  $0.00625$ & $2$ & $2$ & $2$ & $2$ & $2$ \\
  \bottomrule
\end{tabular}

}
\end{table}

$\;$

\null

The next test case is inspired by the one proposed by Sundar \textit{et al.} in \cite{burstedde2012}.
The problem is defined in the {$\Omega = (-1, 1) \times (-1, 1)$} square,
with homogeneous Dirichlet BCs and $f = 1$ on the whole $\Omega$.
The grid step $h$ is a fixed $h = 2^{-7}$.
The $\mu$ function is discontinuous, and has the checkerboard pattern shown below;
on white tiles $\mu = 1$ while on black tiles $\mu = 10^{7}$.
The main focus of this test is to make sure that the solver mantains a good behaviour when the frequency $K$
of the tiles
becomes higher, \textit{i.e.} the number of discontinuities in the permeability grows.

In the table we compare the iterations needed by the GMRES when solving the preconditioner with a
direct method ($\text{It}_{\text{d}}$), and when applying the multigrid method ($\text{It}_{\text{mg}}$).
\begin{table}
\small
\begin{center}
    \begin{tabular}{cccccc}
        \toprule
        & & & & &\\[-2ex] %hack per lasciare un po' di spazio
        $\mu$  &
    \includegraphics[scale=0.5]{images/burst_6.png}
        &
    \includegraphics[scale=0.5]{images/burst_5.png}
        &
    \includegraphics[scale=0.5]{images/burst_4.png}
        &
    \includegraphics[scale=0.5]{images/burst_3.png}
        &
    \includegraphics[scale=0.5]{images/burst_2.png}
    \\ \midrule
        $K$        & 6 & 5 & 4 & 3 & 2 \\% \hline
    $\text{It}_{\text{d}}$   & 8 & 8 & 8 & 8 & 8 \\% \hline
    $\text{It}_{\text{mg}}$  & 14& 16& 18& 19& 15\\% \hline
    \bottomrule
    \end{tabular}
\end{center}
\end{table}

As one might expected, the multigrid method performs generally worse than the
preconditioner solved with a direct method. However, neither method is
affected by the change in frequency of the checkerboard pattern.

A similar test has been run by stretching the mesh along one spatial dimension
to simulate anisotropic coefficients. The preconditioner
still showed an almost-optimal performance.
%We are more than satisfied by the robustness of our preconditioner, which proved
%to mantain an optimal behaviour even when a discontinuous and highly variable permeability is used.
\end{block}

\end{column} % End of third column

\begin{column}{\sepwid}\end{column} % Empty spacer column

\begin{column}{\onecolwid} % The third column

%----------------------------------------------------------------------------------------
% CONCLUSION
%----------------------------------------------------------------------------------------

\begin{block}{Conclusion}

The numerical performance of the ideal preconditioner with internal direct
solver is as good as we expected: in all our test cases it proved to be robust
to mesh refinement, highly variable coefficients, non-squared domains with
local refinement and anisotropic meshes.

The numerical performance of the ideal preconditioner with internal
multigrid solver was satisfactory as well: even though the number of the outer
GMRES iterations is always higher than its direct counterpart, the number of
iterations stays constant regardless of the level of mesh refinement,
permeability coefficient and grid anisotropy.

\end{block}

\begin{block}{Future developments}

Nunc tempus venenatis facilisis. \textbf{Curabitur suscipit} consequat eros non
porttitor. Sed a massa dolor, id ornare enim. Fusce quis massa dictum tortor
\textbf{tincidunt mattis}. Donec quam est, lobortis quis pretium at, laoreet
scelerisque lacus. Nam quis odio enim, in molestie libero. Vivamus cursus mi at
\textit{nulla elementum sollicitudin}.

\end{block}

%----------------------------------------------------------------------------------------
% REFERENCES
%----------------------------------------------------------------------------------------

\begin{block}{References}

\nocite{*} % Insert publications even if they are not cited in the poster
\small{\bibliographystyle{unsrt}
\bibliography{main}\vspace{0.75in}}

\end{block}

%----------------------------------------------------------------------------------------
% ACKNOWLEDGEMENTS
%----------------------------------------------------------------------------------------

\setbeamercolor{block title}{fg=red,bg=white} % Change the block title color

\begin{block}{Acknowledgements}

\small{\rmfamily{We would like to thank Francesco Cattoglio for the substantial
contributions to the C++ codebase used in this work and his willingness to help.
}} \\

\end{block}

%----------------------------------------------------------------------------------------
% CONTACT INFORMATION
%----------------------------------------------------------------------------------------

\setbeamercolor{block alerted title}{fg=black,bg=norange} % Change the alert block title colors
\setbeamercolor{block alerted body}{fg=black,bg=white} % Change the alert block body colors

\begin{alertblock}{Contact Information}

\begin{itemize}
\item Mattia Penati: \href{mailto:mattia.penati@polimi.it}{mattia.penati@polimi.it}
\item Edie Miglio: \href{mailto:edie.miglio@polimi.it}{edie.miglio@polimi.it}
\item Francesco Cattoglio: \href{mailto:francesco.cattoglio@mail.polimi.it}{francesco.cattoglio@mail.polimi.it}

\hrulefill

The poster can be downloaded from
\begin{center}
    \small
    \url{https://bitbucket.org/mattiapenati/pasc14-poster}
\end{center}
\end{itemize}

\end{alertblock}

\begin{center}
\begin{tabular}{ccc}
    \includegraphics[width=0.4\linewidth]{images/logopoli} %
    & \hfill & %
    \includegraphics[width=0.4\linewidth]{images/logomox}
\end{tabular}
\end{center}

%----------------------------------------------------------------------------------------

\end{column} % End of the third column

\end{columns} % End of all the columns in the poster

\end{frame} % End of the enclosing frame

\end{document}
